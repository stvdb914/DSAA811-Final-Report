% Options for packages loaded elsewhere
\PassOptionsToPackage{unicode}{hyperref}
\PassOptionsToPackage{hyphens}{url}
%
\documentclass[
]{article}
\usepackage{amsmath,amssymb}
\usepackage{iftex}
\ifPDFTeX
  \usepackage[T1]{fontenc}
  \usepackage[utf8]{inputenc}
  \usepackage{textcomp} % provide euro and other symbols
\else % if luatex or xetex
  \usepackage{unicode-math} % this also loads fontspec
  \defaultfontfeatures{Scale=MatchLowercase}
  \defaultfontfeatures[\rmfamily]{Ligatures=TeX,Scale=1}
\fi
\usepackage{lmodern}
\ifPDFTeX\else
  % xetex/luatex font selection
\fi
% Use upquote if available, for straight quotes in verbatim environments
\IfFileExists{upquote.sty}{\usepackage{upquote}}{}
\IfFileExists{microtype.sty}{% use microtype if available
  \usepackage[]{microtype}
  \UseMicrotypeSet[protrusion]{basicmath} % disable protrusion for tt fonts
}{}
\makeatletter
\@ifundefined{KOMAClassName}{% if non-KOMA class
  \IfFileExists{parskip.sty}{%
    \usepackage{parskip}
  }{% else
    \setlength{\parindent}{0pt}
    \setlength{\parskip}{6pt plus 2pt minus 1pt}}
}{% if KOMA class
  \KOMAoptions{parskip=half}}
\makeatother
\usepackage{xcolor}
\usepackage[margin= 2.00cm]{geometry}
\usepackage{color}
\usepackage{fancyvrb}
\newcommand{\VerbBar}{|}
\newcommand{\VERB}{\Verb[commandchars=\\\{\}]}
\DefineVerbatimEnvironment{Highlighting}{Verbatim}{commandchars=\\\{\}}
% Add ',fontsize=\small' for more characters per line
\usepackage{framed}
\definecolor{shadecolor}{RGB}{248,248,248}
\newenvironment{Shaded}{\begin{snugshade}}{\end{snugshade}}
\newcommand{\AlertTok}[1]{\textcolor[rgb]{0.94,0.16,0.16}{#1}}
\newcommand{\AnnotationTok}[1]{\textcolor[rgb]{0.56,0.35,0.01}{\textbf{\textit{#1}}}}
\newcommand{\AttributeTok}[1]{\textcolor[rgb]{0.13,0.29,0.53}{#1}}
\newcommand{\BaseNTok}[1]{\textcolor[rgb]{0.00,0.00,0.81}{#1}}
\newcommand{\BuiltInTok}[1]{#1}
\newcommand{\CharTok}[1]{\textcolor[rgb]{0.31,0.60,0.02}{#1}}
\newcommand{\CommentTok}[1]{\textcolor[rgb]{0.56,0.35,0.01}{\textit{#1}}}
\newcommand{\CommentVarTok}[1]{\textcolor[rgb]{0.56,0.35,0.01}{\textbf{\textit{#1}}}}
\newcommand{\ConstantTok}[1]{\textcolor[rgb]{0.56,0.35,0.01}{#1}}
\newcommand{\ControlFlowTok}[1]{\textcolor[rgb]{0.13,0.29,0.53}{\textbf{#1}}}
\newcommand{\DataTypeTok}[1]{\textcolor[rgb]{0.13,0.29,0.53}{#1}}
\newcommand{\DecValTok}[1]{\textcolor[rgb]{0.00,0.00,0.81}{#1}}
\newcommand{\DocumentationTok}[1]{\textcolor[rgb]{0.56,0.35,0.01}{\textbf{\textit{#1}}}}
\newcommand{\ErrorTok}[1]{\textcolor[rgb]{0.64,0.00,0.00}{\textbf{#1}}}
\newcommand{\ExtensionTok}[1]{#1}
\newcommand{\FloatTok}[1]{\textcolor[rgb]{0.00,0.00,0.81}{#1}}
\newcommand{\FunctionTok}[1]{\textcolor[rgb]{0.13,0.29,0.53}{\textbf{#1}}}
\newcommand{\ImportTok}[1]{#1}
\newcommand{\InformationTok}[1]{\textcolor[rgb]{0.56,0.35,0.01}{\textbf{\textit{#1}}}}
\newcommand{\KeywordTok}[1]{\textcolor[rgb]{0.13,0.29,0.53}{\textbf{#1}}}
\newcommand{\NormalTok}[1]{#1}
\newcommand{\OperatorTok}[1]{\textcolor[rgb]{0.81,0.36,0.00}{\textbf{#1}}}
\newcommand{\OtherTok}[1]{\textcolor[rgb]{0.56,0.35,0.01}{#1}}
\newcommand{\PreprocessorTok}[1]{\textcolor[rgb]{0.56,0.35,0.01}{\textit{#1}}}
\newcommand{\RegionMarkerTok}[1]{#1}
\newcommand{\SpecialCharTok}[1]{\textcolor[rgb]{0.81,0.36,0.00}{\textbf{#1}}}
\newcommand{\SpecialStringTok}[1]{\textcolor[rgb]{0.31,0.60,0.02}{#1}}
\newcommand{\StringTok}[1]{\textcolor[rgb]{0.31,0.60,0.02}{#1}}
\newcommand{\VariableTok}[1]{\textcolor[rgb]{0.00,0.00,0.00}{#1}}
\newcommand{\VerbatimStringTok}[1]{\textcolor[rgb]{0.31,0.60,0.02}{#1}}
\newcommand{\WarningTok}[1]{\textcolor[rgb]{0.56,0.35,0.01}{\textbf{\textit{#1}}}}
\usepackage{longtable,booktabs,array}
\usepackage{calc} % for calculating minipage widths
% Correct order of tables after \paragraph or \subparagraph
\usepackage{etoolbox}
\makeatletter
\patchcmd\longtable{\par}{\if@noskipsec\mbox{}\fi\par}{}{}
\makeatother
% Allow footnotes in longtable head/foot
\IfFileExists{footnotehyper.sty}{\usepackage{footnotehyper}}{\usepackage{footnote}}
\makesavenoteenv{longtable}
\usepackage{graphicx}
\makeatletter
\newsavebox\pandoc@box
\newcommand*\pandocbounded[1]{% scales image to fit in text height/width
  \sbox\pandoc@box{#1}%
  \Gscale@div\@tempa{\textheight}{\dimexpr\ht\pandoc@box+\dp\pandoc@box\relax}%
  \Gscale@div\@tempb{\linewidth}{\wd\pandoc@box}%
  \ifdim\@tempb\p@<\@tempa\p@\let\@tempa\@tempb\fi% select the smaller of both
  \ifdim\@tempa\p@<\p@\scalebox{\@tempa}{\usebox\pandoc@box}%
  \else\usebox{\pandoc@box}%
  \fi%
}
% Set default figure placement to htbp
\def\fps@figure{htbp}
\makeatother
\setlength{\emergencystretch}{3em} % prevent overfull lines
\providecommand{\tightlist}{%
  \setlength{\itemsep}{0pt}\setlength{\parskip}{0pt}}
\setcounter{secnumdepth}{5}
% definitions for citeproc citations
\NewDocumentCommand\citeproctext{}{}
\NewDocumentCommand\citeproc{mm}{%
  \begingroup\def\citeproctext{#2}\cite{#1}\endgroup}
\makeatletter
 % allow citations to break across lines
 \let\@cite@ofmt\@firstofone
 % avoid brackets around text for \cite:
 \def\@biblabel#1{}
 \def\@cite#1#2{{#1\if@tempswa , #2\fi}}
\makeatother
\newlength{\cslhangindent}
\setlength{\cslhangindent}{1.5em}
\newlength{\csllabelwidth}
\setlength{\csllabelwidth}{3em}
\newenvironment{CSLReferences}[2] % #1 hanging-indent, #2 entry-spacing
 {\begin{list}{}{%
  \setlength{\itemindent}{0pt}
  \setlength{\leftmargin}{0pt}
  \setlength{\parsep}{0pt}
  % turn on hanging indent if param 1 is 1
  \ifodd #1
   \setlength{\leftmargin}{\cslhangindent}
   \setlength{\itemindent}{-1\cslhangindent}
  \fi
  % set entry spacing
  \setlength{\itemsep}{#2\baselineskip}}}
 {\end{list}}
\usepackage{calc}
\newcommand{\CSLBlock}[1]{\hfill\break\parbox[t]{\linewidth}{\strut\ignorespaces#1\strut}}
\newcommand{\CSLLeftMargin}[1]{\parbox[t]{\csllabelwidth}{\strut#1\strut}}
\newcommand{\CSLRightInline}[1]{\parbox[t]{\linewidth - \csllabelwidth}{\strut#1\strut}}
\newcommand{\CSLIndent}[1]{\hspace{\cslhangindent}#1}
\usepackage{setspace}
\onehalfspacing
\usepackage{fancyhdr}
\pagestyle{fancy}
\lhead{\fontsize{11pt}{11pt}\selectfont "Times New Roman"}
\fancyhead[L]{\fontsize{11}{11}\selectfont STAT950/STAT250 ASSESSMENT 03 - Sharon Van Den Berg 9251936}
\fancyhead[R]{}
\lfoot{\fontsize{11pt}{11pt}\selectfont "Serif"}
\fancyfoot[C]{\thepage}
\fancyfoot[L]{}
\usepackage{pdflscape}
\newcommand{\blandscape}{\begin{landscape}}
\newcommand{\elandscape}{\end{landscape}}
\usepackage{booktabs}
\usepackage{longtable}
\usepackage{array}
\usepackage{multirow}
\usepackage{wrapfig}
\usepackage{float}
\usepackage{colortbl}
\usepackage{pdflscape}
\usepackage{tabu}
\usepackage{threeparttable}
\usepackage{threeparttablex}
\usepackage[normalem]{ulem}
\usepackage{makecell}
\usepackage{xcolor}
\usepackage{bookmark}
\IfFileExists{xurl.sty}{\usepackage{xurl}}{} % add URL line breaks if available
\urlstyle{same}
\hypersetup{
  pdftitle={Assignment 3 - Final Report},
  pdfauthor={Sharon Van Den Berg},
  hidelinks,
  pdfcreator={LaTeX via pandoc}}

\title{Assignment 3 - Final Report}
\usepackage{etoolbox}
\makeatletter
\providecommand{\subtitle}[1]{% add subtitle to \maketitle
  \apptocmd{\@title}{\par {\large #1 \par}}{}{}
}
\makeatother
\subtitle{Setting the environment}
\author{Sharon Van Den Berg}
\date{2025-03-24}

\begin{document}
\maketitle

{
\setcounter{tocdepth}{2}
\tableofcontents
}
\href{https://github.com/stvdb914/DSAA811-Preliminary-Report}{This report was made using R Markdown as a project linking to my Git account}

\newpage

\section{Abstract}\label{abstract}

\begin{itemize}
\tightlist
\item
  For now, just a heading for this section
\end{itemize}

\section{Glossary}\label{glossary}

\begin{longtable}[l]{>{}ll}
\toprule
Acronym & Definition\\
\midrule
\textbf{\cellcolor{gray!10}{DUMMY}} & \cellcolor{gray!10}{Testing to see if two rows ccan be printed}\\
\textbf{NOC} & National Olympic Committee\\
\bottomrule
\end{longtable}

  %Display the table of contents and bypass the R commands to do this
  \newpage
  \tableofcontents
  \newpage

\section{Introduction}\label{introduction}

\begin{itemize}
\tightlist
\item
  For now, just a heading for this section
\end{itemize}

\section{Background}\label{background}

I have been looking into some articles regarding the statistics around reducing the number of athletes that a country can send to the Olympics to procure the maximum medals. The following three articles are of interest, and depending on the information, I can extract from my dataset. One of these avenues will become the focus element for this report.

(\citeproc{ref-haut_2014_nothing}{Haut, Prohl, and Emrich 2014}) examines the statistics of the Olympics but also investigates the data from the perspective of increasing funds to rural areas to enhance performance and ultimately achieve more winners. At this stage I don't think that this report will be useful, however there is some regional information in my data set that I can use to plot the locations of the athletes on a world map.

(\citeproc{ref-condon_1999_predicting}{Condon, Golden, and Wasil 1999}) Uses neural networks to produce three models that look at winners from a country perspective using data up until 1996. If I can get the same results, my data set continues to 2016, so this avenue would allow me to extend their findings on the years post 1996 to 2016.

(\citeproc{ref-heazlewood_2006_prediction}{Heazlewood 2006}) Looks into creating models to predict the optimal athlete numbers for all swimming events. This article was able to make some of these predictions, but improvements are needed to apply these results to athletics and swimming across the various distances of the races. These models are applied to results from 2004 and earlier. This article is of particular interest because of the break down of athletes into sections, mainly swimming. It could be useful to attempt to extend their results to apply it to other sports, such as athletics in summer or cross country skiing in winter.

Some of the terminology that I have come across is NOC or National Olympic Committee. Each country has a member that is on the committee and they are responsible for that country. Every athlete must compete under their NOC. I can used this NOC to get a general location of the athletes home town when comparing countries for athlete levels.

\section{Research questions and aims of the project}\label{research-questions-and-aims-of-the-project}

As this is in the early stages of investigation, I still have too many questions to look into.

I can look at athletes, per winter or summer games, per country that the athlete competes at, or per NOC. I can even split the data into per sport. At the end of the day I want to answer, what is the minimal number of athletes that my country should send to the games to receive the most medals based on previous summer and winter Olympic data.

My plan moving forward is to play with the data set some more to see if I can marry the results of prior written research so I can stand on the giants of the past to push the research in this area forward in the future.

\section{Rationale}\label{rationale}

It is no real stretch to underestimate the importance of pride that can come from winning many medals at an Olympic games. From the eyes of the country the cost to participate can be exorbitant to send one athlete, let alone an entire team of athletes. The rational for this project is to maximize the number of medals that a country can win, whilst reducing the costs of sending athletes to perform on this stage. I am looking for the optimal number of competing athletes to maximize the gold. In order to look into this problem, we can use past results in order to predict the future.
I am unsure at this stage if we can look at this in the scope of the entire country or if we can reduce this to certain sporting events, such as swimming or track and field teams.

\section{Data Description}\label{data-description}

\begin{center}\includegraphics{C:\Final DSAA811 Report\DSAA811-Final-Report\report\DSAA811_Final_files/figure-latex/SummerPlayers-1} \end{center}

The (\citeproc{ref-bansal_2021_olympics_}{Bansal 2021}) data set called ``Olympics\_'' was compiled
by ``Harsh Bansal'' and was last updated 4 years ago. The dataset was
uploaded and sourced from Kaggle (\citeproc{ref-keating_2025_kaggle}{Keating et al. 2025}). According to
the site, there is only one owner with no DOI Citation, provenance or
license. The restriction on the data is placed on it by Kaggle by way of
citation of the owner ``Harsh Bansal''. I am using this data at my own
risk as it has not been authenticated or carefully curated.

The dataset contains 4 files, ``athlete\_events\_data\_dictionary.csv''
contains 15 observations of 2 variables, ``country\_definitions.csv''
contains 230 observations of 3 variables,
``country\_definitions\_data\_dictionary.csv'' contains 3 observations of 2
variables, and ``athlete\_events.csv'' containing 271,116 observations of
15 variables.

The ``athlete\_events.csv'' file contains all athlete information of all
the Olympic games dating from 1896 summer games and 1924 winter games up
to and including the 2016 summer Olympic games. The following table
outlines the variables contained within the set.

\begin{tabular}{l|l}
\hline
Field & Description\\
\hline
ID & Unique number for each athlete\\
\hline
Name & Athlete's name\\
\hline
Sex & Male (M) or Female (F)\\
\hline
Age & Integer\\
\hline
Height & In centimeters\\
\hline
Weight & In kilograms\\
\hline
Team & Team name\\
\hline
NOC & National Olympic Committee 3-letter code\\
\hline
Games & Year and season\\
\hline
Year & Integer\\
\hline
Season & Summer or Winter\\
\hline
City & Host city\\
\hline
Sport & Sport\\
\hline
Event & Event\\
\hline
Medal & Gold, Silver, Bronze, or NA\\
\hline
\end{tabular}

The variables that I am most interested in is the medal type, so as a
country we can maximize receiving these. The country that the athlete is
from so we can gain counts of participants in each prior games. This
will allow us to work out the proportion of winners. The sport they
participated in to break down the best results. Potentially the height
and weight for some sports are equally important. This information will
become clearer as further graphs and analysis is performed during the
next 7 weeks.

In the athletes table there is a field called NOC which is the National
Olympic City code that links to the country definitions that will allow
for better groupings of data by country when linked to each other.

\section{Exploritory data analysis}\label{exploritory-data-analysis}

The first thing we should do with the datasets is to load them into r using the following code.

\begin{Shaded}
\begin{Highlighting}[]
\CommentTok{\#Read in the 4 csv files}
\NormalTok{athletes }\OtherTok{\textless{}{-}} \FunctionTok{read.csv}\NormalTok{(}\StringTok{\textquotesingle{}./data/athlete\_events\_data\_dictionary.csv\textquotesingle{}}\NormalTok{, }
                     \AttributeTok{header =} \ConstantTok{TRUE}\NormalTok{)}
\NormalTok{events }\OtherTok{\textless{}{-}} \FunctionTok{read.csv}\NormalTok{(}\StringTok{\textquotesingle{}./data/athlete\_events.csv\textquotesingle{}}\NormalTok{, }
                   \AttributeTok{header =} \ConstantTok{TRUE}\NormalTok{)}
\NormalTok{countryDefdd}\OtherTok{\textless{}{-}} \FunctionTok{read.csv}\NormalTok{(}\StringTok{\textquotesingle{}./data/country\_definitions\_data\_dictionary.csv\textquotesingle{}}\NormalTok{,}
                        \AttributeTok{header =} \ConstantTok{TRUE}\NormalTok{)}
\NormalTok{countryDef }\OtherTok{\textless{}{-}} \FunctionTok{read.csv}\NormalTok{(}\StringTok{\textquotesingle{}./data/country\_definitions.csv\textquotesingle{}}\NormalTok{,}
                       \AttributeTok{header =} \ConstantTok{TRUE}\NormalTok{)}

\CommentTok{\#Get a summary view of the data}
\FunctionTok{summary}\NormalTok{(athletes)}
\FunctionTok{summary}\NormalTok{(events)}
\end{Highlighting}
\end{Shaded}

The athletes table is the meta data for the events table. There is a lot of missing data in the events table for height and weight of the athletes. NOC, sex, and year are categorical variables and have been coded as characters or numerical. These will need to be re coded into factors.

\begin{Shaded}
\begin{Highlighting}[]
\FunctionTok{summary}\NormalTok{(countryDefdd)}
\end{Highlighting}
\end{Shaded}

\begin{verbatim}
##     Field           Description       
##  Length:3           Length:3          
##  Class :character   Class :character  
##  Mode  :character   Mode  :character
\end{verbatim}

\begin{Shaded}
\begin{Highlighting}[]
\FunctionTok{summary}\NormalTok{(countryDef)}
\end{Highlighting}
\end{Shaded}

\begin{verbatim}
##      NOC               region             notes          
##  Length:230         Length:230         Length:230        
##  Class :character   Class :character   Class :character  
##  Mode  :character   Mode  :character   Mode  :character
\end{verbatim}

CountryDefdd is the meta data for the countryDef file. These columns are NOC, the region that can be used for geospatial maps and the actual country name if the geospace location is unavailable. The countryDef is the data to represent this information.

Before I try to perform some explorations on the data it is imperative that we clean the data up a bit. Factoring the above variables will help with speed to process the data.

\begin{Shaded}
\begin{Highlighting}[]
\NormalTok{events}\SpecialCharTok{$}\NormalTok{Sex }\OtherTok{\textless{}{-}} \FunctionTok{factor}\NormalTok{(events}\SpecialCharTok{$}\NormalTok{Sex,}
                     \AttributeTok{levels =} \FunctionTok{c}\NormalTok{(}\StringTok{"M"}\NormalTok{,}\StringTok{"F"}\NormalTok{),}
                     \AttributeTok{labels =} \FunctionTok{c}\NormalTok{(}\StringTok{"M"}\NormalTok{,}\StringTok{"F"}\NormalTok{))}

\CommentTok{\#years \textless{}{-} distinct(events, Year)}
\CommentTok{\#events$Year \textless{}{-} factor(events$Year,}
  \CommentTok{\#                    levels = years,}
  \CommentTok{\#                    labels = years)}
\CommentTok{\#class(events$Year)}
\end{Highlighting}
\end{Shaded}

From here we can get a breakdown of the number of athletes that compete in each sport since the 2000 Summer games as shown below.

\begin{Shaded}
\begin{Highlighting}[]
\NormalTok{Summer }\OtherTok{\textless{}{-}}\NormalTok{ events }\SpecialCharTok{\%\textgreater{}\%} \FunctionTok{filter}\NormalTok{ (Season }\SpecialCharTok{==} \StringTok{"Summer"}\NormalTok{) }\SpecialCharTok{\%\textgreater{}\%} \FunctionTok{filter}\NormalTok{ (Year }\SpecialCharTok{\textgreater{}} \DecValTok{2000}\NormalTok{)}
\end{Highlighting}
\end{Shaded}

\begin{Shaded}
\begin{Highlighting}[]
\NormalTok{txtTitle }\OtherTok{\textless{}{-}} \FunctionTok{paste}\NormalTok{(}\StringTok{\textquotesingle{}Number of athletes per sport, per year, between 2000 \textquotesingle{}}\NormalTok{,}
        \StringTok{\textquotesingle{}and 2020 at the summer olympic games\textquotesingle{}}\NormalTok{)}
\NormalTok{Summer }\SpecialCharTok{\%\textgreater{}\%} 
  \FunctionTok{ggplot}\NormalTok{() }\SpecialCharTok{+}
  \FunctionTok{geom\_bar}\NormalTok{(}\FunctionTok{aes}\NormalTok{(}\AttributeTok{y =} \FunctionTok{fct\_rev}\NormalTok{(}\FunctionTok{fct\_infreq}\NormalTok{(Sport))), }\AttributeTok{stat=}\StringTok{"count"}\NormalTok{) }\SpecialCharTok{+}
  \FunctionTok{labs}\NormalTok{(}\AttributeTok{title =}\NormalTok{ txtTitle, }
       \AttributeTok{x =} \StringTok{"Number of athletes"}\NormalTok{, }\AttributeTok{y =} \StringTok{"Sport"}\NormalTok{) }\SpecialCharTok{+} 
  \FunctionTok{theme\_bw}\NormalTok{() }\SpecialCharTok{+} 
  \FunctionTok{theme}\NormalTok{(}\AttributeTok{axis.text =} \FunctionTok{element\_text}\NormalTok{(}\AttributeTok{family=}\StringTok{"serif"}\NormalTok{, }\AttributeTok{size =} \DecValTok{14}\NormalTok{)) }\SpecialCharTok{+}  
  \FunctionTok{facet\_wrap}\NormalTok{(}\FunctionTok{vars}\NormalTok{(Summer}\SpecialCharTok{$}\NormalTok{Year))}
\end{Highlighting}
\end{Shaded}

\pandocbounded{\includegraphics[keepaspectratio]{C:/Final DSAA811 Report/DSAA811-Final-Report/report/DSAA811_Final_files/figure-latex/summergraph-1.pdf}}
Similarly we can get a break down of the number of competing athletes at the Winter olympic games since 2000

\begin{Shaded}
\begin{Highlighting}[]
\NormalTok{txtTitle }\OtherTok{\textless{}{-}} \FunctionTok{paste}\NormalTok{(}\StringTok{\textquotesingle{}Number of athletes per sport, per year,\textquotesingle{}}\NormalTok{,}
       \StringTok{\textquotesingle{} between 2000 and 2020 at the winter olympic games\textquotesingle{}}\NormalTok{)}
\NormalTok{winter }\SpecialCharTok{\%\textgreater{}\%} 
  \FunctionTok{ggplot}\NormalTok{() }\SpecialCharTok{+}
  \FunctionTok{geom\_bar}\NormalTok{(}\FunctionTok{aes}\NormalTok{(}\AttributeTok{y =} \FunctionTok{fct\_rev}\NormalTok{(}\FunctionTok{fct\_infreq}\NormalTok{(Sport))), }\AttributeTok{stat=}\StringTok{"count"}\NormalTok{) }\SpecialCharTok{+}
  \FunctionTok{labs}\NormalTok{(}\AttributeTok{title =}\NormalTok{ txtTitle, }
       \AttributeTok{x =} \StringTok{"Number of athletes"}\NormalTok{, }\AttributeTok{y =} \StringTok{"Sport"}\NormalTok{) }\SpecialCharTok{+} 
  \FunctionTok{theme\_bw}\NormalTok{() }\SpecialCharTok{+} 
  \FunctionTok{theme}\NormalTok{(}\AttributeTok{axis.text =} \FunctionTok{element\_text}\NormalTok{(}\AttributeTok{family=}\StringTok{"serif"}\NormalTok{, }\AttributeTok{size =} \DecValTok{14}\NormalTok{)) }\SpecialCharTok{+} 
  \FunctionTok{facet\_wrap}\NormalTok{(}\FunctionTok{vars}\NormalTok{(winter}\SpecialCharTok{$}\NormalTok{Year))}
\end{Highlighting}
\end{Shaded}

\pandocbounded{\includegraphics[keepaspectratio]{C:/Final DSAA811 Report/DSAA811-Final-Report/report/DSAA811_Final_files/figure-latex/wintergraph-1.pdf}}

Whilst these plots look cool it is hard to read. It does take up a lot of space on the page, and I am not seeing any major changes to the number of athletes that have been sent in the past four summer or winter olympics. If this data is needed a table would be a better representation of it. Below is the same information for one of the olympics, summer or winter.

\begin{Shaded}
\begin{Highlighting}[]
\NormalTok{summerCounts }\OtherTok{\textless{}{-}}\NormalTok{ Summer }\SpecialCharTok{\%\textgreater{}\%} 
  \FunctionTok{group\_by}\NormalTok{ (Year,Sport) }\SpecialCharTok{\%\textgreater{}\%}
  \FunctionTok{count}\NormalTok{(Sport) }\SpecialCharTok{\%\textgreater{}\%} 
  \FunctionTok{arrange}\NormalTok{(Year, }\FunctionTok{desc}\NormalTok{(n)) }\SpecialCharTok{\%\textgreater{}\%}
  \FunctionTok{pivot\_wider}\NormalTok{(}\AttributeTok{values\_from =}\NormalTok{ n, }\AttributeTok{names\_from =}\NormalTok{ Sport)}

\NormalTok{summerCounts }\OtherTok{\textless{}{-}} \FunctionTok{rotate\_df}\NormalTok{(summerCounts)}
\FunctionTok{print.data.frame}\NormalTok{ (summerCounts)}
\end{Highlighting}
\end{Shaded}

\begin{verbatim}
##                         V1   V2   V3   V4
## Year                  2004 2008 2012 2016
## Athletics             2175 2244 2278 2508
## Swimming              1618 1749 1538 1568
## Gymnastics            1151  996  848  861
## Cycling                623  652  629  667
## Rowing                 559  563  550  550
## Shooting               558  577  560  555
## Canoeing               434  436  418  441
## Football               425  469  467  473
## Sailing                401  400  379  380
## Judo                   384  386  384  389
## Equestrianism          371  340  350  355
## Hockey                 352  387  387  390
## Wrestling              342  343  339  346
## Handball               328  343  347  353
## Fencing                323  332  345  346
## Basketball             287  287  287  281
## Volleyball             283  283  287  283
## Boxing                 280  283  283  283
## Table Tennis           260  250  236  236
## Water Polo             255  256  257  258
## Tennis                 252  254  286  286
## Weightlifting          249  253  252  255
## Archery                212  194  200  200
## Badminton              200  184  182  177
## Diving                 196  182  181  178
## Baseball               191  191   NA   NA
## Taekwondo              124  126  128  126
## Softball               118  120   NA   NA
## Synchronized Swimming  117  117  117  118
## Triathlon               99  110  110  110
## Beach Volleyball        96   96   96   96
## Rhythmic Gymnastics     84   95   95   96
## Modern Pentathlon       64   72   72   72
## Trampolining            32   32   32   32
## Rugby Sevens            NA   NA   NA  299
## Golf                    NA   NA   NA  120
\end{verbatim}

\begin{Shaded}
\begin{Highlighting}[]
\NormalTok{winterCounts }\OtherTok{\textless{}{-}}\NormalTok{ winter }\SpecialCharTok{\%\textgreater{}\%} 
  \FunctionTok{group\_by}\NormalTok{ (Year, Sport) }\SpecialCharTok{\%\textgreater{}\%}
  \FunctionTok{count}\NormalTok{(Sport) }\SpecialCharTok{\%\textgreater{}\%} 
  \FunctionTok{arrange}\NormalTok{(Year, }\FunctionTok{desc}\NormalTok{(n)) }\SpecialCharTok{\%\textgreater{}\%}
  \FunctionTok{pivot\_wider}\NormalTok{(}\AttributeTok{values\_from =}\NormalTok{ n, }\AttributeTok{names\_from =}\NormalTok{ Sport)}

\NormalTok{winterCounts }\OtherTok{\textless{}{-}} \FunctionTok{rotate\_df}\NormalTok{(winterCounts)}
\FunctionTok{print.data.frame}\NormalTok{ (winterCounts)}
\end{Highlighting}
\end{Shaded}

\begin{verbatim}
##                             V1   V2   V3   V4
## Year                      2002 2006 2010 2014
## Cross Country Skiing       774  812  725  765
## Biathlon                   564  658  683  726
## Alpine Skiing              559  619  685  687
## Ice Hockey                 468  442  420  443
## Speed Skating              332  379  366  368
## Short Track Speed Skating  251  239  273  269
## Bobsleigh                  240  191  199  223
## Ski Jumping                178  202  170  200
## Figure Skating             143  147  146  221
## Nordic Combined            125  138  131  128
## Snowboarding               118  198  185  308
## Luge                       117  108  107  156
## Freestyle Skiing           105  116  172  264
## Curling                     96   91   93   87
## Skeleton                    39   42   47   46
\end{verbatim}

This data set lets you plot the athletes height and weight to see if there is any benifit to selecting athletes of a certain structure. Below is an example using the swimming data in the data set

\begin{Shaded}
\begin{Highlighting}[]
\NormalTok{  events }\SpecialCharTok{\%\textgreater{}\%} \FunctionTok{filter}\NormalTok{(Sport }\SpecialCharTok{==} \StringTok{"Swimming"}\NormalTok{) }\SpecialCharTok{\%\textgreater{}\%}
  \FunctionTok{ggplot}\NormalTok{() }\SpecialCharTok{+}
  \FunctionTok{geom\_point}\NormalTok{(}\FunctionTok{aes}\NormalTok{(}\AttributeTok{x =}\NormalTok{ Age, }\AttributeTok{y =}\NormalTok{ Weight)) }\SpecialCharTok{+}
  \FunctionTok{labs}\NormalTok{(}\AttributeTok{title =} \StringTok{\textquotesingle{}Olympic swimmers ages and weights\textquotesingle{}}\NormalTok{) }\SpecialCharTok{+}
  \FunctionTok{theme\_bw}\NormalTok{()}
\end{Highlighting}
\end{Shaded}

\begin{verbatim}
## Warning: Removed 4411 rows containing missing values or values outside the scale range
## (`geom_point()`).
\end{verbatim}

\pandocbounded{\includegraphics[keepaspectratio]{C:/Final DSAA811 Report/DSAA811-Final-Report/report/DSAA811_Final_files/figure-latex/numericals-1.pdf}}
From this we can see that swimmers are optimal when they are under the age of 40 and smaller than 100 kilograms in general, across all the Olympics.

At this stage I still need to find some more research articles that I can replicate to make a story with this data set, There are still many avenues that I can follow up with.

\section{Conclusion / Discussion}\label{conclusion-discussion}

\newpage

\#Session Information

\begin{Shaded}
\begin{Highlighting}[]
\FunctionTok{sessionInfo}\NormalTok{()}
\end{Highlighting}
\end{Shaded}

\begin{verbatim}
## R version 4.3.2 (2023-10-31 ucrt)
## Platform: x86_64-w64-mingw32/x64 (64-bit)
## Running under: Windows 11 x64 (build 26100)
## 
## Matrix products: default
## 
## 
## locale:
## [1] LC_COLLATE=English_United States.utf8 
## [2] LC_CTYPE=English_United States.utf8   
## [3] LC_MONETARY=English_United States.utf8
## [4] LC_NUMERIC=C                          
## [5] LC_TIME=English_United States.utf8    
## 
## time zone: Australia/Sydney
## tzcode source: internal
## 
## attached base packages:
## [1] stats     graphics  grDevices utils     datasets  methods   base     
## 
## other attached packages:
##  [1] magrittr_2.0.3   maps_3.4.2.1     sjmisc_2.8.10    kableExtra_1.4.0
##  [5] sf_1.0-20        lmerTest_3.1-3   lme4_1.1-37      Matrix_1.6-5    
##  [9] lubridate_1.9.4  forcats_1.0.0    stringr_1.5.1    purrr_1.0.4     
## [13] readr_2.1.5      tibble_3.2.1     tidyverse_2.0.0  ggplot2_3.5.2   
## [17] dplyr_1.1.4      tidyr_1.3.1      tinytex_0.57     knitr_1.50      
## 
## loaded via a namespace (and not attached):
##  [1] gtable_0.3.6        xfun_0.51           insight_1.2.0      
##  [4] lattice_0.22-7      tzdb_0.5.0          numDeriv_2016.8-1.1
##  [7] vctrs_0.6.5         tools_4.3.2         Rdpack_2.6.4       
## [10] generics_0.1.4      proxy_0.4-27        pkgconfig_2.0.3    
## [13] KernSmooth_2.23-26  RColorBrewer_1.1-3  lifecycle_1.0.4    
## [16] compiler_4.3.2      farver_2.1.2        htmltools_0.5.8.1  
## [19] class_7.3-23        yaml_2.3.10         pillar_1.10.2      
## [22] nloptr_2.2.1        MASS_7.3-60         classInt_0.4-11    
## [25] reformulas_0.4.1    boot_1.3-31         nlme_3.1-168       
## [28] sjlabelled_1.2.0    tidyselect_1.2.1    digest_0.6.37      
## [31] stringi_1.8.7       bookdown_0.43       labeling_0.4.3     
## [34] splines_4.3.2       rprojroot_2.0.4     fastmap_1.2.0      
## [37] grid_4.3.2          cli_3.6.2           dichromat_2.0-0.1  
## [40] e1071_1.7-16        withr_3.0.2         scales_1.4.0       
## [43] timechange_0.3.0    rmarkdown_2.29      hms_1.1.3          
## [46] evaluate_1.0.3      rbibutils_2.3       viridisLite_0.4.2  
## [49] rlang_1.1.5         Rcpp_1.0.14         glue_1.8.0         
## [52] DBI_1.2.3           xml2_1.3.8          svglite_2.1.3      
## [55] rstudioapi_0.17.1   minqa_1.2.8         R6_2.6.1           
## [58] systemfonts_1.2.2   units_0.8-7
\end{verbatim}

\newpage

\section*{Bibliography}\label{bibliography}
\addcontentsline{toc}{section}{Bibliography}

\phantomsection\label{refs}
\begin{CSLReferences}{1}{0}
\bibitem[\citeproctext]{ref-bansal_2021_olympics_}
Bansal, Harsh. 2021. {``Olympics\_.''} Kaggle.com. \url{https://www.kaggle.com/datasets/harshbansal27/olympics}.

\bibitem[\citeproctext]{ref-condon_1999_predicting}
Condon, Edward M, Bruce L Golden, and Edward A Wasil. 1999. {``Predicting the Success of Nations at the Summer Olympics Using Neural Networks.''} \emph{Computers \& Operations Research} 26 (November): 1243--65. \url{https://doi.org/10.1016/s0305-0548(99)00003-9}.

\bibitem[\citeproctext]{ref-haut_2014_nothing}
Haut, Jan, Robert Prohl, and Eike Emrich. 2014. {``Nothing but Medals? Attitudes Towards the Importance of Olympic Success.''} \emph{International Review for the Sociology of Sport} 51 (March): 332--48. \url{https://doi.org/10.1177/1012690214526400}.

\bibitem[\citeproctext]{ref-heazlewood_2006_prediction}
Heazlewood, Timothy. 2006. {``Prediction Versus Reality: The Use of Mathematical Models to Predict Elite Performance in Swimming and Athletics at the Olympic Games.''} \emph{Journal of Sports Science \& Medicine} 5 (December): 480. \url{https://pmc.ncbi.nlm.nih.gov/articles/PMC3861753/}.

\bibitem[\citeproctext]{ref-keating_2025_kaggle}
Keating, Nate, Jeff Moser, William Cukierski, Jerad Rose, Myles O'Neill, Risdal Meg, Meghan O'Connell, et al. 2025. {``Kaggle: Your Home for Data Science.''} Kaggle.com. \url{https://www.kaggle.com/}.

\end{CSLReferences}

\end{document}
